%%=============================================================================
%% Samenvatting
%%=============================================================================

% TODO: De "abstract" of samenvatting is een kernachtige (~ 1 blz. voor een
% thesis) synthese van het document.
%
% Deze aspecten moeten zeker aan bod komen:
% - Context: waarom is dit werk belangrijk?
% - Nood: waarom moest dit onderzocht worden?
% - Taak: wat heb je precies gedaan?
% - Object: wat staat in dit document geschreven?
% - Resultaat: wat was het resultaat?
% - Conclusie: wat is/zijn de belangrijkste conclusie(s)?
% - Perspectief: blijven er nog vragen open die in de toekomst nog kunnen
%    onderzocht worden? Wat is een mogelijk vervolg voor jouw onderzoek?
%
% LET OP! Een samenvatting is GEEN voorwoord!

%%---------- Nederlandse samenvatting -----------------------------------------

\IfLanguageName{english}{%
\selectlanguage{dutch}
\chapter*{Samenvatting}

\selectlanguage{english}
}{}

%%---------- Samenvatting -----------------------------------------------------
% De samenvatting in de hoofdtaal van het document

\chapter*{\IfLanguageName{dutch}{Samenvatting}{Abstract}}

Dit onderzoek bespreekt de 4 belangrijkste open source logging oplossingen welke gebruikt kunnen worden bij het ontwikkelen van applicaties gebaseerd op een microservice architectuur. Dit onderzoek neemt de omgeving van Be-Mobile als voorbeeld en referentiepunt. Hier is sprake van een Kubernetes cluster met 150+ nodes. Na de recente release van Loki door Grafana moet een vergelijkende studie tussen Loki en de al wat meer gevestigde logging oplossingen gevoerd worden. 

Het is de bedoeling van dit onderzoek om bedrijven die willen overschakelen naar een microservice omgeving te helpen met de zoektocht naar een geschikte logging oplossing. Ook bedrijven die reeds functioneren in een microservice omgeving hebben baat bij het lezen van dit onderzoek. Het kan meer duidelijkheid scheppen over de verschillen tussen elke oplossing. Aangezien drie van de vier onderzochte oplossingen gebruik maken van Elasticsearch kan afgevraagd worden wat het verschil hiertussen deze is.

Het onderzoek focust zich vooral op enkele kernpunten die elke goede logging oplossing moet bevatten. Ook wordt voor elke oplossing getoond hoe de installatie van elke oplossing wordt uitgevoerd. Verder wordt de theorie uitgelegd om deze ook op Kubernetes te installeren en te configureren. Uit dit onderzoek komt naar voren dat er geen duidelijke meest geschikte open source logging oplossing is. Loki kan als meest geschikte aanschouwd worden maar deze oplossing is nog steeds in ontwikkeling en zal pas binnen enkele maanden in productie gebruikt kunnen worden. Verder maken alle andere oplossingen gebruik van Elasticsearch wat voor redelijk wat geheugenverbruik zorgt. De meeste geschikte oplossing hierbij is de EFK stack. De conclusies die uit dit onderzoek getrokken worden leiden tot de vraag naar dieper onderzoek met een grotere testomgeving. Verder zou een extra onderzoek naar Loki binnen enkele maanden aan de orde zijn.
