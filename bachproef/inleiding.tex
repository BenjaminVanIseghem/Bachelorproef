%%=============================================================================
%% Inleiding
%%=============================================================================

\chapter{Inleiding}
\label{ch:inleiding}

Alle logs op een centrale plaats hebben is een belangrijk aspect bij elke applicatie. Bij monolithische architecturen is dit vanzelfsprekend, maar dankzij de opkomst van de microservice architectuur is dit niet langer mogelijk zonder hulp van tools. Dit onderzoek bestudeert het verschil tussen een monolithische en een microservice architectuur. Daarna wordt op zoek gegaan naar de meest gepaste logging oplossing voor een microservice architectuur. In dit onderzoek komen enkel de open-source oplossingen aanbod zodat deze volledig zelf kunnen gehost worden en er dus geen extra kosten aan te pas komen voor cloud services. 

\section{Probleemstelling}
\label{sec:probleemstelling}

Go wordt steeds meer gebruikt en wordt door sommigen geprezen als een ideale taal voor microservices. Door de gedecentraliseerde structuur van microservices wordt het moeilijk om het overzicht te bewaren bij het debuggen van applicaties. Elke service heeft een eigen console met logs. Steeds opnieuw wisselen van console en scrollen naar een specifieke log is zeer inefficiënt. Waar naar op zoek moet gegaan worden is een gecentraliseerde logging oplossing die overzicht biedt aan al uw applicaties. De doelgroep voor dit onderzoek is bedrijven die in het proces zijn van overstappen van een monolithische naar een microservice architectuur of reeds overgestapt zijn en nog op zoek zijn naar een oplossing voor gecentralizeerde logging. In dit werk zullen meerdere oplossingen aangeboden worden en zal er aan de hand van een lijst requirements vergeleken welke het meest geschikt is om in productie uit te rollen.

\section{Onderzoeksvraag}
\label{sec:onderzoeksvraag}

\begin{itemize}
    \item  Wat is de beste open-source logging met tracing oplossing voor een bedrijf als Be-Mobile?
\end{itemize}

\section{Onderzoeksdoelstelling}
\label{sec:onderzoeksdoelstelling}

Het is de bedoeling van dit onderzoek om een vergelijking te maken tussen de ondersteunde logging platformen in Go. Er zal een afweging gemaakt worden voor de impact die de oplossing heeft op het systeem, de installatie, het gebruiksgemak, en in welke mate er support is van de community. Een algemene oplossing zal niet geboden worden in dit onderzoek. Wel zal er een opsomming gebeuren van de mogelijkheden en zal aan elke oplossing een score gegeven worden aan de hand van enkele requirements.

\begin{itemize}
    \item Installatie 
    \item Gebruiksgemak
    \item Impact op het systeem
    \item Support door community
\end{itemize}

\section{Opzet van deze bachelorproef}
\label{sec:opzet-bachelorproef}

De rest van deze bachelorproef is als volgt opgebouwd:

In Hoofdstuk~\ref{ch:stand-van-zaken} wordt een overzicht gegeven van de stand van zaken binnen het onderzoeksdomein, op basis van een literatuurstudie.

In Hoofdstuk~\ref{ch:methodologie} wordt de methodologie toegelicht. 

In Hoofdstuk~\ref{ch:ELK} wordt de installatie en configuratie van de ELK of Elastic stack toegelicht. Verder worden alle requirements voor een logging solution vergeleken met de mogelijkheden van de ELK of Elastic stack en een score gegeven voor elke requirement.

In Hoofdstuk~\ref{ch:EFK} wordt de installatie en configuratie van de EFK stack toegelicht. Verder worden alle requirements voor een logging solution vergeleken met de mogelijkheden van de EFK stack en een score gegeven voor elke requirement.

In Hoofdstuk~\ref{ch:graylog} wordt de installatie en configuratie van Graylog toegelicht. Verder worden alle requirements voor een logging solution vergeleken met de mogelijkheden van Graylog en een score gegeven voor elke requirement.

In Hoofdstuk~\ref{ch:loki} wordt de installatie en configuratie van Loki en Promtail toegelicht. Verder worden alle requirements voor een logging solution vergeleken met de mogelijkheden van Loki en een score gegeven voor elke requirement.

In Hoofdstuk~\ref{ch:conclusie}, tenslotte, wordt de conclusie gegeven en een antwoord geformuleerd op de onderzoeksvragen. Daarbij wordt ook een aanzet gegeven voor toekomstig onderzoek binnen dit domein.

