%%=============================================================================
%% Conclusie
%%=============================================================================

\chapter{Conclusie}
\label{ch:conclusie}

In dit onderzoek wordt een antwoord gegeven op de onderzoeksvraag `Wat is de beste open-source logging met tracing oplossing voor een bedrijf als Be-Mobile?`. Om dit te onderzoeken is een vergelijkende studie uitgevoerd tussen de voornaamste vier logging oplossingen, namelijk de Elastic stack, de ELK stack, Graylog, en Grafana Loki. Er werd gekeken naar enkele kernelementen van een goede logging oplossing en elk van de onderzochte oplossingen werd hieraan getoetst.

Als eerste conclusie kan gesteld worden dat logging en tracing verschillende onderwerpen zijn. Tracing werd ook onderzocht in dit werk en hoewel dit onderzoek zeer nuttig is geweest voor Be-Mobile is er besloten om de uitwerking hiervan niet op te nemen in dit werk.

Een tweede conclusie is dat er geen enkele oplossing voldoet aan de eisen die in het begin van dit onderzoek gesteld werden. EFK en ELK scoren uitstekend op visualisatie en documentatie, maar door de overhead die gepaard gaat met Elasticsearch zijn deze minder geschikte oplossingen. De complexiteit van de manier waarop Elasticsearch geconfigureerd moet worden in een Kubernetes cluster speelt hier een grote rol. Loki daarentegen zorgt voor een minimale overhead dankzij de architectuur ervan en biedt veel mogelijkheden. Het nadeel aan Loki is dat ontwikkeling ervan zich nog steeds in een beta fase bevindt. Dit leidt tot een gebrek aan documentatie en slechts een minimale ondersteuning in Grafana. Ondanks de jeugdigheid van Loki, kan het nu al aanschouwt worden als de meest geschikte oplossing in een microservice omgeving met een hoog aantal nodes. Waar EFK, ELK, en Graylog reeds volledig ontwikkeld zijn en dus vastzitten aan de manier waarop ze zijn opgebouwd, is Loki nog volop in ontwikkeling en bezit het dus veel potentieel om uit te groeien tot de ideale open source logging oplossing op de markt.

Een derde conclusie die kan getrokken worden uit dit onderzoek is dat elk van de onderzochte oplossingen beter gehost wordt door de service die de deze ontwikkeld heeft. De kost die gepaard gaat met de uitbreiding van een Kubernetes cluster zal op een gegeven moment de kost van hosting voorbij gaan. Het nadeel aan deze aanpak is het verlies van controle over de cluster. Hoewel dit zeker geen oplossing is in het geval van Be-Mobile, kan dit zeker een optie zijn voor een ander bedrijf die dit werkt leest.

Concreet kan gesteld worden dat de EFK stack de voordeligste optie is wanneer Elasticsearch wordt gebruikt. Loki is de oplossing met het meeste potentieel. De keuze hiertussen moet gemaakt worden op basis van budget. Wanneer beschikt wordt over een groot budget, zal EFK de meest geschikte oplossing zijn. Indien hier absoluut geen budget voor beschikbaar is, moet geïnvesteerd worden in de toekomst met Loki als meest geschikte oplosssing.

Deze conclusies waren volledig onverwacht. Zoals te zien is in het onderzoeksvoorstel van dit werk, is te zien dat oorspronkelijk tracing verwerkt was in dit werk. Ook werd verwacht dat op zoek zou gegaan worden naar een databank en logging framework, niet naar een volledige oplossing. Bij aanvang van het onderzoek zelf werd snel duidelijk dat het werk zou gaan over de verschillende logging oplossingen. Zelfs op dat moment werden zulke conclusies niet verwacht. 

Wegens de nogal opiniegerichte wijze van vergelijken in dit werk, wordt de nood aan een nieuw onderzoek duidelijk. Dit zou dieper ingaan op de impact op het systeem van elke oplossing. Een simulatie van een soortgelijke omgeving als bij Be-Mobile is hiervoor vereist. Deze vergelijking zou in een latere fase nogmaals uitgevoerd kunnen worden, dit wanneer Loki al wat verder ontwikkeld werd. De verwachte conclusie voor die vergelijking zou dan zeker in het verdeel van Loki zijn.

Dit werk kan als nuttig beschouwd worden door bedrijven of startups die op zoek zijn naar een logging backend voor hun microservices. De requirement `Moet kunnen scalen naargelang de groeiende Kubernetes cluster` is een belangrijke requirement voor een zulke doelgroep. Ook personen die interesse in microservices of logging in het algemeen kunnen dit werk gebruiken als basis om een diepere kennis te vergaren in het onderwerp.